% Sample file for 'CACM - Research Highlights'-type articles
% Created by: Gerry Murray, Elec. Pub. Info. Mgr., Pubs. Dept., ACM HQ, NY.
% (murray@hq.acm.org)
%
% This is "research-highlights-sample.tex" (sample file) V1.1 Sept. 2008
% This file should be compiled with V1.1 of "research4cacm.cls" Sept. 2008
%
% If you have already submitted an article to an ACM/SIGS Conference, and have had your
% article published in one of the 'Proceedings', then you have probably used
% the (ACM/SIGS) 'sig-alternate' class and .tex file.
% Any such 'conference-prepared' source .tex file is 'compatible' with the class file
% 'research4cacm' which you will use to prepare your article for inclusion in the magazine 'CACM'.
%
% Here are the steps to take in order to 'morph' your article from being
% a 'conference' article to one more suitable for inclusion in 'Communications of the ACM'.
%
% 1. Change \documentclass{sig-alternate}  to \documentclass{research4cacm}
%
% 2. Comment out the conference information e.g.  %\conferenceinfo{WOODSTOCK}{'97 El Paso, Texas USA}
%
% 3. Make sure the copyright data is correct e.g. \crdata{0001-0782/08/0200}
%
% 4. Make sure the YEAR is correct e.g. \CopyrightYear{2008} with the current default being �2008�.
%
% 5. Comment out the Classification scheme, general terms and keywords.
%
% 6. If you have mentioned authors in the 'Additional Authors' section you should
%    'move them back' into the byline (so that they appear with all the other authors).
%    ALL authors, in Research Highlights articles, get equal billing.
%
% 7. Suitably edit the file (i.e. body text) so as to make it more appropriate for a wider audience.
%
% If, early on in the editorial process, the 'correct' copyright data becomes available
% please change the copyright data to suit, otherwise leave the default '0001-0782/08/0X00'.
% (The editorial staff will change it later on in the production cycle.)
%
% ================= IF YOU HAVE QUESTIONS =======================
% Technical questions _only_ to
% Gerald Murray (murray@hq.acm.org)
% ===============================================================
% ---------------------------------------------------------------------------------------------------------------
%
\documentclass{research4cacm}

\begin{document}
%
% --- Author Metadata here ---
% Conference information is NOT appropriate for CACM so comment it out.
%\CopyrightYear{2008} % Allows default copyright year (2008) to be over-ridden - IF NEED BE.
%\crdata{0001-0782/08/0X00}  % Allows default copyright data (0001-0782/08/0X00) to be over-ridden - IF NEED BE.
% --- End of Author Metadata ---

\title{Reproducibility Studies in Computational Sciences
% \titlenote{(This is a simple titlenote.)For use with research4cacm.cls. Supported by ACM.}
%
% Show use of \thanks - which can appear here (normal/default) or down by the author
% \thanks{The original version of this paper is entitled ``XXX" and was
% published in (Title of publication, publication date, publisher.)}
}

%\subtitle{[Extended Abstract]
%\titlenote{A full version of this paper is available in...}
%}
%
% You need the command \numberofauthors to handle the 'placement
% and alignment' of the authors beneath the title.
%
% For aesthetic reasons, we recommend 'three authors at a time'
% i.e. three 'name/affiliation blocks' be placed beneath the title.
%
% NOTE: You are NOT restricted in how many 'rows' of
% "name/affiliations" may appear. We just ask that you restrict
% the number of 'columns' to three.
%
% Use the \alignauthor commands to handle the names
% and affiliations.
%
\numberofauthors{2}
%
\author{
% You can go ahead and credit any number of authors here,
% e.g. one 'row of three' or two rows (consisting of one row of three
% and a second row of one, two or three).
%
% The command \alignauthor (no curly braces needed) should
% precede each author name, affiliation/snail-mail address and
% e-mail address. Additionally, tag each line of
% affiliation/address with \affaddr, and tag the
% e-mail address with \email.
%
% 1st. author
\alignauthor
Tom Crick\\
       \affaddr{Department of Computing}\\
       \affaddr{Cardiff Metropolitan University}\\
       \affaddr{Cardiff, UK}\\
       \email{tcrick@cardiffmet.ac.uk}
% 2nd. author
\alignauthor
Ian P. Gent\\
       \affaddr{School of Computer Science}\\
       \affaddr{University of St Andrews}\\
       \affaddr{Fife, UK}\\
       \email{ian.gent@st-andrews.ac.uk}
}

\maketitle
\begin{abstract}
Reproducibility is a key tenet of all modern sciences.  Unfortunately
the current state of reproducibility in computational sciences is
lamentable.  We propose that journals in computational sciences should
encourage papers reproducing (or failing to reproduce) earlier work in
their field, by establishing clear criteria for assessing
reproducibility papers.
\end{abstract}

% The classification Scheme, General Terms and Keywords are not appropriate for CACM so comment them out.

\section*{Notes}

Not sure about the name "reproducibility papers" because "reproducible
papers" is a name for, e.g., Sweave'd papers.

We should look through journal instructions to authors for mention of
reproducibility both on output (ie. new papers should ahve some nod
towards reproducibility) and more importantly on input (are
reproducibility studies explicitly encouraged.)  Probably best to
archive them and put the details online, and put data in the
paper. Should be sensible and reasonable sized sample (e.g. all ACM
journals, journals in some other sensible ranking).

(cite to check bib~\cite{crick-et-al_wssspe2})

\section*{Bullet Points}

\begin{itemize}


\item initiatives in other disciplines (Cardiff person?) pre-registration?
\item Positive results may be unpublishable!   

\item Reproducibility

\item Culture Change 
\item Incentives.  

\item Feeling a fraud as open stuff... true across the field.  E.g. we trust resutls from others and reproduce things.

\item cf dissemination model is broken... impending crisis

\item E.g. give examples of ourselves.   ICLP 09?  Ian can find lots, honest!

\item Examples of tracks applications ... 

\item part of core phd skills training
\item instead of giving a student 100 papers to read and write a literature review, give them one and write a reproducibility studies
\item thinking of Ian Miguel and Lars Kotthoff, got journal papers out of lit reviews
\item Ian's first paper about counterfactuals





\item Example of good journals
\begin{itemize}
\item Springer?
\item IPOL
\item
Journal 2010-2012
Never published a paper!
Folded into Code 
http://www.scfbm.org/content/pdf/1751-0473-7-2.pdf

\item reproducibility project in psychology?

\item thing where people reproduced old papers.  (repo 

\end{itemize}



\end{itemize}

\section{ Review standards }

Make a particular proposal. 

Reproducibility is broad ... 


Quality of work and paper is AIJ quality
Novelty not assessed in normal way
Originality is in study of reproducibility of some major piece of AI
Could expose flaws in conventional wisdom if original experiments not reproducible
Could show that conventional wisdom is completely correct 
Either way community has solid basis for understanding 
Often expect new critical insights e.g. on importance of neglected implementation issues
Paper itself should have very high standard of reproducibility
Compare survey paper:
Novelty not assessed in normal way
Likely to improve understanding through selection, presentation, and new insights

\section{What's next?}

Post this to lots of people


%ACKNOWLEDGMENTS are optional
%\section{Acknowledgments}
% This section is optional; it is a location for you
% to acknowledge grants, funding, editing assistance, collaboration, access to
% uncommon, but necessary, hardware or software etc.
% In the present case, for example, the
% authors would like to thank Gerald Murray of ACM for
% his help in codifying this \textit{Author's Guide}
% and the \textbf{.cls} and \textbf{.tex} files that it describes.


% The following two commands are all you need in the
% initial runs of your .tex file to
% produce the bibliography for the citations in your paper.
\bibliographystyle{abbrv} % standard abbrv style
\bibliography{reproducibilityCACM}  % substitute the name of 'your' Bibliography here
% You must have a proper ".bib" file
% and remember to run:
% latex bibtex latex latex (in that particular order) in order to resolve all the 'numerical values'
% be they for figures, equation numbers, references, footnotes, etc. etc.
%
%\balancecolumns
% That's all folks! % GM Sept. 2008
\end{document}
